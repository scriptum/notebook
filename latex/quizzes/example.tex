% Пример теста с двумя вариантами по языку Bash. Внимательно читайте 
% комментарии к вопросам и командам!
% Полная версия (с обложкой) пока не переведена

\documentclass[12pt,a4paper]{article}

% блок поддержки русского языка
\usepackage[T1,T2A]{fontenc}
\usepackage[utf8]{inputenc}
\usepackage[russian]{babel} % для корректной работы переносов строк
\usepackage[a4paper]{geometry} % у американцев по умолчанию letter

% микротипография - красивое уплотнение текста
\usepackage[activate={true,nocompatibility},
    final,
    tracking=true,kerning=true,spacing=true,
    factor=1100,
    stretch=10,
    shrink=10]{microtype}

% \sloppy % полностью выключить выход за поля
\tolerance=800 % не вылезать за поля по возможности

% \usepackage{amsmath}
% \usepackage{graphicx}

\usepackage[
  % НЕ ТРОГАТЬ, автоматически (рас)комментируется в gen.sh
  % nosolutions,
  answerkey, % раскрыть и убрать nosolutions - получим вариант с ответами
  %
  % vspacewithsolns, % ещё один вариант - печатать лист с ответами в конце
  % coverpage,coverpagesumry=byparts, % обложка (TODO)
    pointsonboth, % сумма баллов для каждого вопроса
    totalsonright, % в конце каждой страницы - сумма баллов
    forpaper,
    useforms
]{eqexam}

% компактный заголовок
\makeatletter
\renewcommand\maketitledesign
{%
\begin{tabular}{p{\textwidth}}
Ф.И.О., группа:~\hrulefill{}~<<\rule{6mm}{0.1pt}>>\rule{3cm}{0.1pt}~\the\year{} \\
\\
\normalsize\textbf{\webtitle\hfill\websubject}
\\
~
\end{tabular}
}
\makeatother

% локализация
\renewcommand\leftmarginPtsTxt[1]{\small($#1~{\text{бал.}}$)}
\renewcommand\leftmarginPtsEaTxt[1]{\small($#1~_{\text{отв.}}^{\text{бал.}}$)}
\renewcommand\marginpointsboxtext[2]{\small$#1\,\text{бал.}$}
\renewcommand\itemPTsTxt[1]{($#1$ бал.)}
\renewcommand\totalsboxtext{\small$\theeqpointsthispage\,\text{бал.}$}

\examNameLabel{Студент:}
% чистим заголовки
\lheadeqe{}
\cheadeqe{--~\arabic{page}~--}
\rheadeqe{}

% название курса
\title[T2]{Программирование Bash}
% автор (не отображается по умолчанию)
\author{П. Г. Рощин}
% номер варианта задания, можно добавить ещё \vC{3} и т.п.
\subject[C1]{Вариант \vA{1}\vB{2}}
% \date{}
% \keywords{Test~1, Section 001}

% количество вариантов заданий
\numVersions{2}

% какой вариант генерим (a, b ...)
\forVersion{a}
% \forVersion{b}

% наименование вуза для полной версии
\university{НИЯУ <<МИФИ>>}
% адрес электронной почты (не используется)
% \email{bash@example.ru}

\begin{document}

% шапка для заполнения ФИО студента, номер варианта
\maketitle

\begin{exam}[Часть I.]{Part1}

% инструкция к экзамену (а кто их читает?)
% \begin{instructions}[Часть I.]
% Solve each of the problems without error. If you make an error,
% points will be subtracted from your total score.
% \end{instructions}

% простой вопрос с произвольным ответом. Варианты скрыты в \vA{} \vB{} и т.п.
% какой у нас вариант - такой и появится блок. За вопрос 9 баллов
\begin{problem}[9]
Это вопрос для \vA{первого}\vB{второго} варианта.
\begin{solution}[2cm]
Это ответ для \vA{первого}\vB{второго} варианта.
\end{solution}
\end{problem}

% вопрос с выбором одного варианта ответа
\begin{problem}[1]
Как узнать время работы системы с момента последней перезагрузки?
    \begin{answers}{4} % 4 колонки в таблице для 4 ответов
    \Ans0 time & \Ans0 datetime & \Ans1 uptime & \Ans0 никак
    \end{answers}
\end{problem}

% вопрос с длинными вариантами ответа
\begin{problem}[1]
Что делает команда \texttt{mkdir example}?
    \begin{answers}{1}
    \Ans0 Создаёт файл с именем example
    \Ans1 Создаёт каталог с именем example % правильный ответ
    \Ans0 Показывает пример использования команды mkdir
    \end{answers}
\end{problem}

% вопрос с длинными вариантами ответа с отличием по вариантам
\begin{problem}[1]
Назначение переменной \texttt{\vA{\$\#}\vB{\$?}} --- это:
    \begin{answers}{1}
    \Ans{\vA{0}\vB{1}} Код возврата последней выполненной команды
    \Ans{\vA{1}\vB{0}} Количество аргументов, переданных сценарию
    \Ans0 Это некорректная конструкция
    \end{answers}
\end{problem}

% вопросы "правда или ложь"
\begin{problem*}[6]
Поставьте <<$+$>> для корректных синтаксических конструкций в bash и <<$-$>>
для некорректных.
\fillinWidth\defaultTFwidth % для аккуратного вида
\begin{parts}
    \item \TF{+} \texttt{[ \$x -\vA{g}\vB{l}t \$y ]}
    \item \TF{--} \texttt{( \$x -\vA{l}\vB{g}t \$y )}
    \item \TF{--} \texttt{\$x -\vA{l}\vB{g}t \$y}
    \item \TF{+} \texttt{test \$x -\vA{g}\vB{l}t \$y}
% два варианта, каждый внутри своего окружения
\vA{
    \item \TF{+} \texttt{for i in a b c; do echo \$i; done}
    \item \TF{--} \texttt{for \char`\"{}\$i\char`\"{} in a b c; do echo \$i; done}
}
\vB{
    \item \TF{--} \texttt{for \$j in a b c; do echo \$j; done}
    \item \TF{+} \texttt{for j in a b c; do echo "\$j"; done}
}
\end{parts}
\end{problem*}

% Заполнение пропущенных слов, для пущей запутанности заставляем компьютер
% самостоятельно суммировать баллы за задание
\begin{problem*}[\auto]
Впишите правильные команды
\fillinWidth\defaultTFwidth % для аккуратного вида
\begin{parts}
    \item \PTs{1} Команда \fillin{7mm}{wc} используется для подсчёта слов.
    \item \PTs{2} Для получения прав на исполнение нового сценария нужно выполнить команду \fillin{20mm}{chmod +x}.
    \item \PTs{1} Команда \fillin{7mm}{ls} отображает содержимое текущего каталога.
    \item \PTs{1} Команда \fillin{10mm}{pwd} выводит путь до текущего каталога.
\end{parts}
\end{problem*}

\end{exam}

\end{document}
